\chapter{Introduction}
\label{chap:intro}

This document is a brief user manual for the RFSM GUI application (in short ``RFSM application'').
The RFSM application aims at describing, drawing and simulating \emph{reactive finite state
  machines}. Reactive FSMs are a FSMs for which transitions can only take place at the occurence of
events.

\medskip
The RFSM application has been developed mainly for pedagogical purposes, in order to initiate students to
model-based design. It is currently used in courses dedicated to embedded system design both on
software and hardware platforms (microcontrolers and FPGA resp.). It can be used to

\begin{itemize}
\item describe FSM-based models and testbenches,
\item generate graphical representations of these models (\verb|.dot| format) for visualisation,
\item simulate these models, producing \verb|.vcd| files to be displayed with waveform viewers such
  as \texttt{gtkwave},
\item generate C, SystemC and VHDL implementations to be integrated to existing applications
  (including testbenches for simulation)
\end{itemize}

\medskip The RFSM application is actually a graphical front-end to the \verb|rfsmc| command-line
compiler\footnote{\texttt{https://github.com/jserot/rfsm}}.  The command-line compiler is described
in a separate document. This document focuses in the GUI. It is organized as follows.
Chapter~\ref{cha:overview} is an informal presentation of the RFSM language and of its possible
usages. Chapter~\ref{cha:gui} describes the GUI-based application. Appendix A gives the detailed
syntax of the language.  Chapter~\ref{cha:overview} and and appendix A are simply copied from the
reference of the command-line compiler. They are only reproduced here for convenience.

\medskip
The RFSM-Light application\footnote{\texttt{https://github.com/jserot/rfsm-light}} is simplified
version of RFSM supporting of purely graphical description of single-FSM systems.

%%% Local Variables: 
%%% mode: latex
%%% TeX-master: "rfsm"
%%% End: 
